\noindent На \pageref{LastPage} с., \totalfigures\ рисунков, \totaltables\ таблиц, 3 приложения

\MakeUppercase{
  ключевые слова: imdf, ios, картография, ортофотоплан, indoor, appclip
}

Тема выпускной квалификационной работы: <<Разработка мобильного навигационного приложения на платформе iOS>>.

Данная работа посвящена разработке мобильного приложения на платформе iOS, отображающего карту помещений и прилегающей к ним территории, а также позволяющего пользователю производить поиск по карте и строить маршруты до интересующих его мест.
В ходе разработки необходимо было решить следующие задачи:
\begin{itemize}
  \item Проанализировать существующие технологии для решения этой проблемы
  \item Разработать методику высоко детализированной картографии и с её помощью составить карту
  \item Разработать мобильное приложение, которое будет отображать карту, предоставлять пользователю поиск и построение маршрута по ней
  \item Разработать возможность "поделиться"\ маршрутом с помощью графических кодов
\end{itemize}

В качестве объекта картографии выбран кампус политехнического университета, составлен его ортофотоплан, по которому проведена картография в формат IMDF. Для отображения карты было разработано мобильное приложение на платформе iOS, поддерживающие все актуальные устройства. В приложение интегрирована технология AppClip, благодаря которой полностью раскрывается возможность поделиться маршрутом.

Разработанное приложение будет актуально для студентов, которые хотят найти кабинет в незнакомом корпусе, а также для организаторов, которые смогут поместить в приглашение ссылку или QR код на маршрут до места проведения мероприятия.
