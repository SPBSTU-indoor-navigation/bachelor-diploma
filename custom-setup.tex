% сюда можно помещать свои настройки не редактируя main, чтобы не решать проблему с мёрджем при изменение шаблона

\usepackage[hidelinks]{hyperref}
\usepackage{pdfpages}
\usepackage[figure,table]{totalcount}

\lstset{ %
  language=swift,               % выбор языка для подсветки (здесь это С)
  basicstyle=\small\sffamily,   % размер и начертание шрифта для подсветки кода
  numbers=left,                 % где поставить нумерацию строк (слева\справа)
  numberstyle=\small,           % размер шрифта для номеров строк
  stepnumber=1,                 % размер шага между двумя номерами строк
  firstnumber=1,
  numberfirstline=true
  numbersep=5pt,                % как далеко отстоят номера строк от подсвечиваемого кода
  backgroundcolor=\color{white},% цвет фона подсветки - используем \usepackage{color}
  showspaces=false,             % показывать или нет пробелы специальными отступами
  showstringspaces=false,       % показывать или нет пробелы в строках
  showtabs=false,               % показывать или нет табуляцию в строках
  frame=single,                 % рисовать рамку вокруг кода
  tabsize=2,                    % размер табуляции по умолчанию равен 2 пробелам
  captionpos=t,                 % позиция заголовка вверху [t] или внизу [b] 
  breaklines=true,              % автоматически переносить строки (да\нет)
  breakatwhitespace=false,      % переносить строки только если есть пробел
  extendedchars=\true,
  escapeinside={\%*}{*)}        % если нужно добавить комментарии в коде
}

\definecolor{delim}{RGB}{20,105,176}
\definecolor{numb}{RGB}{106, 109, 32}
\definecolor{string}{rgb}{0.64,0.08,0.08}

\lstdefinelanguage{json}{
numbers=left,
numberstyle=\small,
frame=single,
rulecolor=\color{black},
showspaces=false,
showtabs=false,
breaklines=true,
postbreak=\raisebox{0ex}[0ex][0ex]{\ensuremath{\color{gray}\hookrightarrow\space}},
breakatwhitespace=true,
basicstyle=\ttfamily\small,
upquote=true,
morestring=[b]",
stringstyle=\color{string},
literate=
*{0}{{{\color{numb}0}}}{1}
{1}{{{\color{numb}1}}}{1}
{2}{{{\color{numb}2}}}{1}
{3}{{{\color{numb}3}}}{1}
{4}{{{\color{numb}4}}}{1}
{5}{{{\color{numb}5}}}{1}
{6}{{{\color{numb}6}}}{1}
{7}{{{\color{numb}7}}}{1}
{8}{{{\color{numb}8}}}{1}
{9}{{{\color{numb}9}}}{1}
{\{}{{{\color{delim}{\{}}}}{1}
{\}}{{{\color{delim}{\}}}}}{1}
{[}{{{\color{delim}{[}}}}{1}
{]}{{{\color{delim}{]}}}}{1},
extendedchars=\true,
escapeend=\end{russian}
}
